\chapter{Wstęp}
W czasie ćwiczenia ... // tu przydałoby się kilka słów wstępu, ale jakoś nie mam weny

\chapter{Implementacja algorytmu DMC}

\chapter{Interfejs użytkownika}
Tworzenie interfejsu rozpoczęliśmy od ustalenia ogólnego układu graficznego. Zdecydowaliśmy się podzielić wyświetlacz na dwie części, z lewej strony wyświatlane są dane dotyczące procesu, a z prawej intuicyjne ``lampki'' sygnalizujące status układu. Poniżej zaprezentowano zdjęcie układu w stanie prawidłowej pracy i z regulacją DMC.


\begin{figure}[ht]
\centering
\includegraphic{4.jpg}
\label{R4}
\end{figure}


Jako sytuację alarmową przyjęto sytuację, w której temperatura T1 znajdzie się poza zakresem od 30 do 50 stopni Celsjusza. Poniżej zaprezentowano wystąpienie takiej sytuacji po ustawieniu temperatury zadanej powyżej tego zakresu. Jest ona wyraźnie sygnalizowana w polu statusu poprzez zmianę ``lampki'' temperatury oraz jej opisu, gdy temperatura jest za wysoka kolor zmienia się na czerwony, a napis na ``T WYSOKA'', gdy temperatura spadnie poniżej 30 stopni ``lampka'' zmieni kolor na niebieski, a napis zmieni się na ``T NISKA''.

\begin{figure}[ht]
\centering
\includegraphic{1.jpg}
\label{R1}
\end{figure}


Podobnie sygnalizowane jest wystąpienie błędu pomiaru.


\begin{figure}[ht]
\centering
\includegraphic{3.jpg}
\label{R3}
\end{figure}


Układ operuje w dwóch trybach: DMC i manualnym, przejście między trybami odbywa się poprzez dotknięcie napisu sygnalizującego obecny stan, wyświetla on odpowiednio napisy ``DMC'' i ``MAN''.


\begin{figure}[ht]
\centering
\includegraphic{2.jpg}
\label{R2}
\end{figure}


Ponieważ wystąpienie błędu w komunikacji sygnalizuje poważny problem --- zgodnie z zaleceniem --- sygnalizujemy go poprzez wyświetlenie dużego napisu na czerwonym tle. Jeżeli problem zniknie samoczynnie to dla informacji operatora układ wyświetla wszystkie dane na czerwonym tle i informuje napisem o wystąpieniu błędu.


\begin{figure}[ht]
\centering
\includegraphic{5.jpg}
\label{R5}
\end{figure}


\begin{figure}[ht]
\centering
\includegraphic{6.jpg}
\label{R6}
\end{figure}

