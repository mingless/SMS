\chapter{Wstęp}
Sprawozdania przygotowywane w~ramach projektów i~ćwiczeń laboratoryjnych muszą być opracowane w~systemie \LaTeX. System składu dokumentów \LaTeX \ jest całkowicie darmowy, ale umożliwia opracowanie bardzo dobrze złożonych dokumentów. Do przygotowania sprawozdania należy wykorzystać klasę \verb+mwrep+ z~pakietu klas \verb+mwcls+ \cite{litWolinski2013} oraz klasę \verb+polski+. W~przypadku dłuższych opracowań (książek, prac dyplomowych) należy wykorzystać klasę \verb+mwbk+.

Jeżeli dostępne są rysunki w~formacie \verb+pdf+, najwygodniej do przetworzenia dokumentu użyć polecenia \verb+pdflatex+, które bezpośrednio generuje dokument w formacie \verb+pdf+. Polecenie \verb+latex+ wymaga rysunków w formacie \verb+ps+ lub \verb+eps+ i~generuje dokument w~formacie \verb+dvi+, który następnie można przekształcić do formatu \verb+eps+ lub \verb+pdf+. Nie używamy rysunków zapisanych w~plikach bitmapowych (\verb+bmp+, \verb+jpg+, \verb+png+). Jedynym wyjątkiem są zdjęcia.

Istnieje wiele podręczników do nauki zasad składania dokumentów w~\LaTeX u, np. doskonała praca zbiorowa \cite{litOetiker2007} lub ew. podręcznik Wikibooks \cite{litlatexwiki2017}. Do edycji dokumentów można wykorzystać np. program \TeX nicCenter, dostępny pod adresem \url{http://www.texniccenter.org}. W~przypadku problemów warto poszukać rozwiązania na forum \url{http://tex.stackexchange.com}.

W~dalszej części dokumentu podano najważniejsze wymagania dotyczące wzorów matematycznych, tabeli i~rysunków. Najszybszą metodą prowadzącą do otrzymania dokumentu jest modyfikacja niniejszego szablonu.